\documentclass[a4j]{jarticle}
\usepackage{graphicx}

\title{情報工学第三 期末レポート}

\author{氏名: 木下直樹\\学籍番号: 09425521}

\date{提出日: 2015月11月26日\\締切日: 2015年11月26日}

\begin{document}
\maketitle

%%%%%%%%%%%%%%%%%%%%%%%%%%%%%%%%%%%%%%%%%%%%%%%
\section{クライアント,サーバモデルの通信の仕組みについて}
%%%%%%%%%%%%%%%%%%%%%%%%%%%%%%%%%%%%%%%%%%%%%%%
クライアントサーバモデルとは, サービスの役割をクライアントとサーバに分離して運用することによってコンピュータの処理を分散する仕組みである. 
サーバはデータベースやサービスを提供するための処理を集中管理し, クライアントは極力データの送受信に専念することで, クライアント側のコンピュータの能力では処理できない,もしくは処理が遅い等の問題が消化される. 

今回の実験では, 通信プロトコルとしてTCP/IPを利用したプログラムを作成した. 以下にTCP/IPでのネットワークの接続やメッセージのやりとりの簡単な流れを使用するシステムコールを示す.

\subsection{クライアントプログラム}
\begin{enumerate}
\item socket()により, サーバと接続するソケットを作成する.
\item ポート番号, IPアドレスを指定するための構造体を設定する.
\item connect()により, サーバとの接続の確立する.
\item send(),recv()によるデータの送受信する.
\item close()でソケットの切断する.
\end{enumerate}

\subsection{サーバプログラム}
\begin{enumerate}
\item socket()により, 接続をリスニングするソケットを作成する.
\item ポート番号, IPアドレスを指定するための構造体を設定する.
\item bind()により, ソケットにポート番号, IPアドレスを設定する.
\item listen()でソケットの接続準備する.
\item accept()でソケットの接続待機する.
\item send(),recv()によるデータの送受信する.
\item close()でソケットの切断する.
\end{enumerate}
%%%%%%%%%%%%%%%%%%%%%%%%%%%%%%%%%%%%%%%%%%%%%%%
\section{自由課題プログラムの作成方針}
%%%%%%%%%%%%%%%%%%%%%%%%%%%%%%%%%%%%%%%%%%%%%%%
今回の自由課題では英単語暗記プログラムを作成する.

具体的には, サーバから単語の日本語訳をクライアントへ送信し, 英語訳をクライアント側の端末で入力してサーバに送信する. サーバはその正誤を判断し, 結果をクライアントへ送るというものだ. 

また, 一連の処理に加え, 問題ファイルを追加する機能と, アカウント機能を追加する. 
このアカウント機能はログインしたクライアントが間違った問題を出力するためのものである.


%%%%%%%%%%%%%%%%%%%%%%%%%%%%%%%%%%%%%%%%%%%%%%%
\section{プログラムについて}
%%%%%%%%%%%%%%%%%%%%%%%%%%%%%%%%%%%%%%%%%%%%%%%
\subsection{プログラムの処理の流れ}
\subsubsection{初期状態の処理}
プログラムを実行すると,次の様にサーバからのメッセージが届く.
\begin{verbatim}
コマンドを入力してください
practice
read
new account
log in
quit
\end{verbatim}
この状態では practice, read, new account, log in, quit の5つのコマンドが入力できる.
それぞれの入力に対する処理をサーバが行う.
また, それ以外の入力をサーバに送信した場合, サーバは上記のメッセージを送信する処理部へループし, クライアント側のプログラムは上記のメッセージを受け取る処理部へループする.

クライアントプログラムの処理の流れは以下である.
\begin{verbatim}
(ループ){
   (ループ){
      (ループ){
         コマンドの入力
         上記の5つのコマンドのいずれかの入力でループから離脱
    }
     read, new account, log inの処理
     コマンド入力時にpractice を入力した場合ループ離脱
   }
   practice の処理
}
\end{verbatim}
read, new account, log in のコマンド処理は一塊のループで行い, practice はそのループの外で処理をする.

サーバプログラムの場合はクライアントからのコマンドを受け取ると, 正しいコマンドである場合その処理をする関数へとび, 正しくないコマンドを受け取った場合, クライアントがループして初期メッセージを受け取れるようにこちらもループをして再度初期メッセージを送信する. ただしquit コマンドを受け取るとクライアントへメッセージを送信することなくプロセスは終了する.

サーバプログラムの処理の流れは以下である.
\begin{verbatim}
(ループ){
   (ループ){
      コマンド入力要求
    正しいコマンドでループ離脱
   }
   各コマンドの処理関数へ入る
}
\end{verbatim}


\subsubsection{log in後の処理}
log in コマンドを実行するとログインフラグの値を変更し, ログイン名を格納する配列にその文字列を格納する. 
また, コマンド入力要求が以下に変わる.
\begin{verbatim}
コマンドを入力してください
practice
read
review
log out
quit
\end{verbatim}
この状態でpractice を実行すると, 間違えた問題がaccountディレクトリ内にあるアカウント名のディレクトリ内のmislog.csvファイルへ出力する(例えば, ./account/naoki/mislog.csv). 
log out で初期状態のコマンド要求部と同じ処理へ戻る.

\subsection{プロトコルについて}
サーバとクライアント間のプロトコルで決められたコマンドとそのコマンドに対するサーバの挙動について説明する.
\begin{itemize}
\item practice\\
  戻り値:[$\backslash$n練習ファイル一覧$\backslash$n(.csvファイル名一覧)]\\
  送られてきた.csvファイル名を選んで入力し, 送信する.
  サーバは送られてきたファイル名が正しければそのファイルを読み込み,
  問の文字列をクライアントへ送信する.
  クライアントはそれに対する答えをサーバへ送信する.
  サーバは結果の正誤と次の問をクライアントへ送信する.
  その送受信を繰り返し. 最後の問の正誤と同時に正解の回答数の結果を送信する.
  また, log in の状態でpractice を実行する時, ログインしたアカウントのディレクトリ内の
  mislog.csv ファイルを開き, 誤回答がある度にその問題と答えをファイルへ出力する.
\item read\\
  戻り値:$[$追加するファイル名を入力してください$]$\\
  クライアントはpracticeで使用したいファイル名をサーバへ送る.
  クライアントプログラムはそのファイルを読み込みサーバへ送信する.
  サーバは受信した文字列をreadディレクトリ内に開いたファイルへ出力する.
  クライアント側にないファイルの名前を送信した場合も正常にファイルが
  読み込めた場合もコマンド要求部へ戻る.
\item new account\\
  戻り値:$[$アカウント名を入力してください$]$\\
  作成したいアカウント名を入力し, 送信する.
  サーバは既にそのアカウントが存在したり, そのアカウントが作成できない場合,
  文字列$[$このアカウントは使えません$]$を送信する.
  アカウントが作成できる場合, 文字列[アカウントを作成しました]を送信する.
  このプログラムでいうアカウントの作成とは, accountディレクトリ内に
  そのアカウント名のディレクトリを作成することである.
\item log in\\
  戻り値:$[$アカウント名を入力してください$]$\\
  クライアントが送信するアカウント名が存在するアカウントの名前でなければ,
  サーバは文字列$[$アカウント名が間違っています$]$をクライアントへ送信する.
  アカウントが存在する場合は, 文字列$[$ログイン:(アカウント名)$]$を
  クライアントへ送信する.
  この際, サーバはログイン状態を示すフラグを更新し, アカウント名を格納する変数に
  そのアカウント名を格納する.
  クライアントプログラムはログインできたことを確認するとサーバ同様ログイン状態を示すフラグを更新する. 
\item log out\\
  戻り値:$[$ログアウトしました$]$\\
  サーバとクライアントのプログラムはログイン状態を示すフラグを更新する.
  また, サーバプログラムはアカウント名変数を初期化する.
\item review\\
  フラグ変数rflagを更新し, practice へ入る.
  review コマンドの入力からpractice へ入る場合, ファイル名の入力要求はされず,
  アカウントディレクトリ内のmislog.csv ファイルを読み込む. 
\item quit\\
  クライアントはこのコマンドを送信し, プログラムを終了する.
  サーバはこのコマンドを受信するとプログラムを終了する. 
\end{itemize}

%%%%%%%%%%%%%%%%%%%%%%%%%%%%%%%%%%%%%%%%%%%%%%%
\section{プログラムの使用法}
%%%%%%%%%%%%%%%%%%%%%%%%%%%%%%%%%%%%%%%%%%%%%%%
\begin{itemize}
\item ログインした場合もしていない場合も問題演習をする場合はpractice コマンドを
入力し, 各問題に対する答えを入力させていく.
\item 演習で間違えた問題だけを集めた演習をしたい時は, あらかじめnew account コマンドを
実行してアカウントを作成しておく.
\item log in コマンドを実行してログインした状態で問題演習をすると間違えた問題が保持され,
review コマンドを実行することでその問題を演習することができる.
\item また, このファイルのデータはプログラムを終了しても
次のpractice の実行(ログイン時)まで保持される.
\item log out コマンドでログアウトができる. アカウントを切り替える際は一度ログアウト
を実行して別のアカウント名でログインするとよい.
\item プログラムを終了する際はquit コマンドを入力する. 
\end{itemize}
%%%%%%%%%%%%%%%%%%%%%%%%%%%%%%%%%%%%%%%%%%%%%%%
\section{プログラムの作成過程と成果物に関する考察}
%%%%%%%%%%%%%%%%%%%%%%%%%%%%%%%%%%%%%%%%%%%%%%%
\subsection{作成過程に関する工夫}
\subsubsection{出力管理について}
本プログラムはデータの送受信をするため, 自身のプログラム自体にエラーがなくコンパイルが通っても
いざ実行すると互いのsendやrecvに関するエラーやバグが発生する. 
そのため, 本プログラムではsendとrecvを関数で管理し, 実行する引数によって送受信した文字列を
txtファイルやターミナルに出力できるようにした. 

ターミナルへの出力に関してはデータの壊れる瞬間や想定していない送受信を感知することに
非常に役立ったが, txtファイルに出力する際は実行を中断するとファイルに正しく書き込まれないことがあったり処理の調整が上手くいかなかったりと, 今回のプログラミングではあまり役にたたなかった. 

ターミナルへ出力する際は元々出力される様になっている文字列がデバック時に邪魔となったので, 
printfも関数やdefineなどを使用して管理し, 出力のONとOFFを切り替え出きるようにすればよかったと思う. 

\subsubsection{バージョン管理について}
本プログラムを作成する際, 小さな改良を目標としたバージョンアップの積み重ねでプログラムを大きくしていくことを心がけた. プログラムを書き換えると高い確立でエラーやバグが発生する. それを改善できればよいが, エラーチェックのための出力や処理の変更により手に負えない状態に陥ることがあるため, プログラムの書き換えをする際にはまず現状のプログラムを別のディレクトリへ退避させ, いつでもその状態へ戻れるようにした. 

常にバックアップをとることでプログラムの書き換えに多少の冒険ができ, 精神的負荷と時間の負荷を軽減することに役立った. しかし, このようなバックアップの管理などはバージョン管理システムなどを利用することが普通であり, それを使いこなせるようになればプログラムを書く際に有利であるため, 次にプログラムを作成する際はこれを利用するべきであると感じた. 

\subsection{成果物に関する工夫}
本プログラムでは機能の幅を広げやすくなるように実行プログラムのあるディレクトリ内のファイルだけでなく, 
別ディレクトリの中を参照したり, ディレクトリを作成したりといった動作を使用した.

例えば, new account コマンドでは, accountディレクトリ内にアカウント名のディレクトリを作成する. 
これにより, 他のアカウントでのログイン時や非ログイン時の操作に影響を与えないエリアを確保できる.
本プログラムでは実装していないが, 例えば以下のような拡張ができる.
\begin{itemize}
\item ログイン時のreadで自分のディレクトリ内にファイルを配置し, practiceで実行プログラムのある
ディレクトリのreadディレクトリ内と自分のディレクトリ内のreadディレクトリ内のファイルを読み込めるようすると, 
個人的な演習ファイルをサーバへ設置することが実現できる. 
\item practice時に保存するファイル名を入力し, review時に演習したいファイルを選択できる.
\end{itemize}

\subsubsection{forkについて}
本プログラムはfork()関数を使うことで複数の端末へ同時にサービスを提供できる. 
サーバプログラムを実行した際はプロセスが1つしか走っていないが, クライアントプログラムとつながるとfork()するためプロセスが2つになる. 一方はクライアントにサービスを提供し, 他方は次のクライアントからの接続要求を待つ.
また, クライアントとの接続が切れたプロセスは終了するようにしている. 
このようにしてサーバは常にクライアントの数より1つ多いプロセスが走行するようにしている.

プロセスを複数走行させる際, それぞれに異なるポートを占有させるようにしている. 4000番を基準にbindできなければ1つ大きいポート番号のbindを試みるというループをし, bindできればループから離脱するというアルゴリズムを採用している. bindできた際にそのポート番号を出力するため, その番号でクライアントはconnectを試みるとよい. 

子プロセスは終了するとプロセスの機能はなくなるが, プロセステーブルには残ってしまう. 子プロセスの生成と終了を繰り返すとゾンビプロセスと呼ばれるこのプロセスが増加し, 一定数に達すると子プロセスが生成できなくなる. そのため, waitpid()システムコールでプロセステーブルから終了した子プロセスの内容を消去する手法を採用した. 

%%%%%%%%%%%%%%%%%%%%%%%%%%%%%%%%%%%%%%%%%%%%%%
\section{実行結果に関する考察}
%%%%%%%%%%%%%%%%%%%%%%%%%%%%%%%%%%%%%%%%%%%%%%%
本プログラムではコマンドを英字で入力することを要求するため, 問いが日本語で答えが英語という形式にしている.
しかし, 問いと答えはカンマ区切りで分けられ, その前後で一方と他方を判別しているだけであり, 問いと答えを反対にしても正誤や問題等の送受信に影響は無いため演習は正常にできる. また, 問いと答えが日本語の問題なども実行できる. 

また, 上記の通り問いと答えはカンマ区切りであり, 一行に問いと答えを収めるように要求しているため, 問題を細かく改行して表示したり, 長すぎる問題は使用できない. 
さらに, カンマを表示させたり複数に区切ることが実装できていない.
よって以下のような仕様の変更案が考えられる.
\begin{itemize}
\item ,,の入力で区切り文字としてでなく出力用のカンマと認識するように変更する\\
  →問題にカンマが使用できるようになる.
\item 波括弧などでくくられた文字列内を一つのまとまりと判断する\\
  →問題にカンマが使用できるほか, 複数の答えの保持などの実装も可能になる
\item 区切り文字がくるまで1行受け取りを続ける\\
  →一行ごとに改行をさせることで問題に改行が実装できるほか, 送受信のデータ容量を気にすること無く問題の容量を増やすことができる\\
  →しかしこれは複数のデータを送るか一つのデータを送るかという判断が必要になるため, 送受信の手順を少し変更しなければならない
\end{itemize}

%%%%%%%%%%%%%%%%%%%%%%%%%%%%%%%%%%%%%%%%%%%%%%%%%%%
\section{プログラムコード}
%%%%%%%%%%%%%%%%%%%%%%%%%%%%%%%%%%%%%%%%%%%%%%%%%%%
クライアントプログラムとサーバプログラムのコードは以下である.
\subsection{クライアントプログラム}
{\baselineskip 2mm
\begin{verbatim}
     1	#include <sys/types.h>
     2	#include <sys/socket.h>
     3	#include <netdb.h>
     4	#include <unistd.h>
     5	#include <stdio.h>
     6	#include <stdlib.h>
     7	#include <string.h>
     8	#include <strings.h>
     9	#include <arpa/inet.h>
    10	
    11	int fd;
    12	char send_buf[1024];
    13	char recv_buf[1024];
    14	int stl;
    15	int i=0;
    16	int printflag=0;
    17	int accountflag=0;
    18	FILE *dfpc;
    19	int rflag=0;
    20	
    21	void check(int x){
    22	  printf("check %d\n",x);
    23	}
    24	
    25	void sendbuf(){
    26	  if(send(fd,send_buf,strlen(send_buf),0)==-1){
    27	    printf("error:send\n");
    28	    exit(0);
    29	  }
    30	  if(printflag==1){
    31	    fprintf(dfpc,"debugs %s\n",send_buf);
    32	    fprintf(dfpc,"================================\n");
    33	  }
    34	  if(printflag==2){
    35	    printf("=============send===================\n");
    36	    printf("%s\n",send_buf);
    37	    printf("=============send===================\n");
    38	  }
    39	  //bzero(send_buf,sizeof(send_buf));
    40	}
    41	
    42	void recvbuf(){
    43	  bzero(recv_buf,sizeof(recv_buf));
    44	  if(recv(fd,recv_buf,1000,0)==-1){
    45	    printf("error:recv\n");
    46	    exit(0);
    47	  }
    48	  if(printflag==1){
    49	    fprintf(dfpc,"debugr %s\n",recv_buf);
    50	    fprintf(dfpc,"********************************\n");
    51	  }
    52	  if(printflag==2){
    53	    printf("************recv*******************\n");
    54	    printf("%s\n",recv_buf);
    55	    printf("************recv*******************\n");
    56	  }
    57	}
    58	
    59	
    60	int subst(char *str, char c1, char c2){
    61	  int n = 0;
    62	  while(*str){
    63	    if(*str == c1){
    64	      *str = c2;
    65	      n++;
    66	    }
    67	    str++;
    68	  }
    69	  return n;
    70	}
    71	
    72	int send_file_data(char *buf){
    73	  FILE *fp;
    74	  
    75	  subst(buf,'\n','\0');
    76	  if((fp = fopen(buf,"r")) == NULL){
    77	    fprintf(stderr,"ファイルがありません\n");
    78	    return 0;
    79	  }
    80	
    81	  sendbuf();
    82	  recvbuf();
    83	  
    84	  while(fgets(send_buf,1000,fp)!=NULL){
    85	    sendbuf();
    86	    printf("%s\n",send_buf);
    87	    recvbuf();
    88	  }
    89	  bzero(send_buf,sizeof(send_buf));
    90	  sprintf(send_buf,"EOF");
    91	  sendbuf();
    92	  bzero(send_buf,sizeof(send_buf));
    93	  fclose(fp);
    94	  return 1;
    95	}
    96	
    97	int start_option(){                         // return 0 で成功
    98	  bzero(send_buf,sizeof(send_buf));
    99	  recvbuf();                      // buf = モードを選んでください
   100	  printf("%s\n",recv_buf);
   101	  read(0,send_buf,1000);
   102	  sendbuf();
   103	  if(strcmp(send_buf,"practice\n")==0) return 0;   // practice で break;
   104	  if(strcmp(send_buf,"quit\n")==0) return -1;
   105	
   106	  else if(accountflag==0){
   107	    if((strcmp(send_buf,"read\n")!=0)&&
   108	       (strcmp(send_buf,"new account\n")!=0)&&
   109	       (strcmp(send_buf,"log in\n")!=0)) return 1;
   110	  }else if(accountflag==1){
   111	    if(strcmp(send_buf,"review\n")==0){
   112	      rflag=1;
   113	      return 0;
   114	    }else if((strcmp(send_buf,"read\n")!=0)&&
   115	       (strcmp(send_buf,"log out\n")!=0)) return 1;
   116	  }
   117	  recvbuf();
   118	  if(strcmp(recv_buf,"追加するファイル名を入力してください")==0){ // read
   119	    printf("%s\n",recv_buf);
   120	    read(0,send_buf,1000);
   121	    if(send_file_data(send_buf)){    // s s
   122	      //      recvbuf();
   123	      return 1;
   124	    }else{
   125	      sprintf(send_buf,"can't open file");
   126	      sendbuf();
   127	      return 1;
   128	    }
   129	  }else if(strcmp(recv_buf,"作成するアカウント名を入力してください")==0){
   130	    printf("%s\n",recv_buf);
   131	    read(0,send_buf,1000);
   132	    sendbuf();
   133	    recvbuf();
   134	    printf("%s\n",recv_buf);
   135	    sprintf(send_buf,"na end");
   136	    sendbuf();
   137	    return 1;
   138	  }else if(strcmp(recv_buf,"アカウント名を入力してください")==0){
   139	    printf("%s\n",recv_buf);
   140	    read(0,send_buf,1000);
   141	    sendbuf();
   142	    recvbuf();
   143	    if(strstr(recv_buf,"ログイン")) accountflag=1;
   144	    printf("%s\n",recv_buf);
   145	    sprintf(send_buf,"login");
   146	    sendbuf();
   147	    return 1;
   148	  }else if(strcmp(recv_buf,"ログアウトしました")==0){
   149	    printf("%s\n",recv_buf);
   150	    accountflag=0;
   151	    sprintf(send_buf,"logout");
   152	    sendbuf();
   153	    return 1;
   154	  }
   155	  return 1;
   156	} 
   157	
   158	int main(int argc, char *argv[]){
   159	  //int fd;
   160	  struct hostent *host;
   161	  struct sockaddr_in sa;
   162	  int pnum=0;
   163	  int i;
   164	  FILE *fp;
   165	  int eflag;
   166	
   167	  if(argc==2) pnum = atoi(argv[1]);
   168	  //printflag = atoi(argv[2]);
   169	
   170	  if((host = gethostbyname("localhost")) == NULL){
   171	    printf("error: gethostbyname\n");
   172	    return 1;
   173	  }
   174	
   175	  if((fd = socket(AF_INET, SOCK_STREAM, 0)) == -1){
   176	    printf("error: socket");
   177	    return 1;
   178	  }
   179	
   180	  sa.sin_family = host->h_addrtype;
   181	  bzero((char *)&sa.sin_addr, 12);
   182	  memcpy((char *)&sa.sin_addr, (char *)host->h_addr, host->h_length);
   183	  sa.sin_port = htons(pnum);
   184	
   185	  if(connect(fd,(struct sockaddr *)&sa, sizeof(sa))==-1){
   186	    printf("error:connect\n");
   187	    return 1;
   188	  }
   189	
   190	  dfpc = fopen("debugc.txt","w");
   191	
   192	  while(1){
   193	    while((eflag=start_option())==1);
   194	    if(eflag==-1)break;
   195	    /* 練習スタート */
   196	      recvbuf();                                      
   197	    if(rflag==0){
   198	      while(1){
   199		printf("%s\n",recv_buf);
   200		stl = read(0,send_buf,1000);
   201		sendbuf();                                     
   202		recvbuf();                                     // buf = "start\n"
   203		if(strcmp(recv_buf,"start\n")==0) break;
   204	      }
   205	    }
   206	    printf("\n\n%s\n",recv_buf);
   207	    sprintf(send_buf,"recvstart\n");
   208	    sendbuf();
   209	    recvbuf();                                     // 1問目
   210	    while(strncmp(recv_buf,"result",6)){
   211	      printf("%s\n",recv_buf);
   212	      bzero(send_buf,sizeof(send_buf));
   213	      stl=read(0,send_buf,1000);
   214	      sendbuf();
   215	      recvbuf();
   216	      printf("%s\n",recv_buf);
   217	      sprintf(send_buf,"next\n");
   218	      sendbuf();
   219	      recvbuf();
   220	    }                                             // recvでループを抜ける
   221	    printf("%s\n",recv_buf);
   222	    sprintf(send_buf,"practice end");
   223	    sendbuf();
   224	    rflag=0;
   225	  }
   226	  fclose(dfpc);  
   227	  if(close(fd) == -1){
   228	    printf("error: close\n");
   229	    return 1;
   230	  }
   231	  
   232	  return 1;
   233	}
\end{verbatim}
}

\subsection{サーバプログラム}
{\baselineskip 2mm
\begin{verbatim}
     1	#include <sys/fcntl.h>
     2	#include <sys/types.h>
     3	#include <sys/socket.h>
     4	#include <netinet/in.h>
     5	#include <netdb.h>
     6	#include <stdio.h>
     7	#include <stdlib.h>
     8	#include <string.h>
     9	#include <ctype.h>
    10	#include <dirent.h>
    11	#include <unistd.h>
    12	#include <errno.h>
    13	#include <sys/wait.h>
    14	#include <signal.h>
    15	
    16	#define DATA_MAX 1024
    17	#define STR_LEN 1024
    18	
    19	int data_count=0;
    20	
    21	char send_buf[STR_LEN + 1];
    22	char recv_buf[STR_LEN + 1];
    23	int sockfd;
    24	int new_sockfd;
    25	int printflag=0;
    26	FILE *dfps;
    27	int accountflag=0;
    28	char accountname[50];
    29	int rflag=0;
    30	
    31	void sendbuf(){
    32	  if(send(new_sockfd, send_buf, strlen(send_buf),0) == -1){
    33	    printf("error : send\n");
    34	    exit(0);
    35	  }
    36	  if(printflag==1){
    37	    fprintf(dfps,"debugs %s\n",send_buf);
    38	    fprintf(dfps,"================================\n");
    39	  }
    40	  if(printflag==2){
    41	    printf("=============send===================\n");
    42	    printf("%s\n",send_buf);
    43	    printf("=============send===================\n");
    44	  }
    45	}
    46	void recvbuf(){
    47	  bzero(recv_buf,sizeof(recv_buf));
    48	  if(recv(new_sockfd,recv_buf,1000,0) == -1){
    49	    printf("error : recv\n");
    50	    exit(0);
    51	  }
    52	  if(printflag==1){
    53	    fprintf(dfps,"debugr %s\n",recv_buf);
    54	    fprintf(dfps,"********************************\n");
    55	  }
    56	  if(printflag==2){
    57	    printf("************recv*******************\n");
    58	    printf("%s\n",recv_buf);
    59	    printf("************recv*******************\n");
    60	  }
    61	}
    62	
    63	void sendrecv(){
    64	  sendbuf();
    65	  recvbuf();
    66	}
    67	
    68	struct tango{
    69	  char japanese[STR_LEN];
    70	  char english[STR_LEN];
    71	};
    72	
    73	struct tango tango[DATA_MAX];
    74	
    75	int subst(char *str, char c1, char c2){
    76	  int n = 0;
    77	  while(*str){
    78	    if(*str == c1){
    79	      *str = c2;
    80	      n++;
    81	    }
    82	    str++;
    83	  }
    84	  return n;
    85	}
    86	
    87	void dent(char *str){
    88	  subst(str,'\n','\0');
    89	}
    90	
    91	void split(char *line,int i){
    92	  int cnt=0;
    93		
    94	  for(cnt=0;*line != ',';cnt++){
    95	    tango[i].japanese[cnt] = *line;
    96	    line++;
    97	  }
    98	  line++;
    99	  for(cnt=0;*line != '\n';cnt++){
   100	    tango[i].english[cnt] = *line;
   101	    line++;
   102	  }
   103	}
   104	
   105	int read_data(char filename[]){
   106	  FILE *fp;
   107	  char line[STR_LEN];
   108	  char str[20];
   109	
   110	  if(rflag==0){
   111	    sprintf(str,"./read/");
   112	    strcat(str,filename);
   113	  }
   114	  else sprintf(str,"./account/%s/mislog.csv",accountname);
   115	
   116	  if((fp = fopen(str,"r"))==NULL){
   117	    fprintf(stderr,"%s\n","error:can't read file.");
   118	    return 0;
   119	  }	
   120	  for(data_count=0;(fgets(line, DATA_MAX + 1, fp) != NULL);data_count++)
   121	    split(line,data_count);
   122	
   123	  fclose(fp);
   124	
   125	  return 1;
   126	}
   127	
   128	void practice(){
   129	  int i;
   130	  int count=0;
   131	  FILE *fp;
   132	  char acfile[20];
   133	
   134	  if(accountflag==1){
   135	    sprintf(acfile,"./account/%s/mislog.csv",accountname);
   136	    if((fp=fopen(acfile,"w"))==NULL){
   137	      printf("error:practice mislog\n");
   138	    }
   139	  }
   140	
   141	  bzero(send_buf,sizeof(send_buf));
   142	  sprintf(send_buf,"start\n");
   143	  sendbuf();
   144	  recvbuf();                      //buf = "ok\n"
   145	  for(i=data_count;i>0;i--){
   146	    sprintf(send_buf,"%s(%d/%d)",
   147		    tango[data_count-i].japanese,
   148		    data_count-i+1, data_count);
   149	    sendbuf();
   150	    recvbuf();                   // ここで解答が送られる
   151	    printf("%s\n",recv_buf);
   152	    dent(recv_buf);
   153	    if(strcmp(recv_buf,tango[data_count-i].english)==0){
   154	      sprintf(send_buf,"correct\n\n");
   155	      count++;
   156	    }else{
   157	      sprintf(send_buf,"incorrect(%s)\n\n",tango[data_count-i].english);
   158	      if(accountflag==1)
   159	      fprintf(fp,"%s,%s\n",
   160		      tango[data_count-i].japanese,tango[data_count-i].english);
   161	    }
   162	    sendbuf();                     // 正誤判定を送信
   163	    recvbuf();                     // buf = "next\n"
   164	    printf("%s\n",recv_buf);
   165	  }
   166	  sprintf(send_buf,"result : %d / %d\n",count,data_count);
   167	  sendbuf();
   168	  recvbuf();
   169	  if(accountflag==1) fclose(fp);
   170	}
   171	
   172	int file_name(char *buf, char dirname[], char ext[]){
   173	  DIR *dirp;
   174	  struct dirent *p;
   175	  char str[1024];
   176	
   177	  bzero(buf,sizeof(buf));
   178		
   179	  if((dirp = opendir(dirname)) == NULL){
   180	    sprintf(buf,"Can't open directory %s\n",dirname);
   181	    return 1;
   182	  }
   183	  while((p = readdir(dirp)) != NULL){
   184	    if(strstr(p->d_name,ext)){
   185	      sprintf(str,"%s ",p->d_name);
   186	      strcat(buf,str);
   187	    }
   188	  }
   189	  strcat(buf,"\n");
   190	  if(closedir(dirp) != 0){
   191	    sprintf(buf,"Can't close directory %s\n",dirname);
   192	    return 1;
   193	  }
   194	  return 0;
   195	}
   196	
   197	int read_file(){
   198	  FILE *fp;
   199	  char file[50];
   200	
   201	  sprintf(file,"./read/");
   202	  sprintf(send_buf,"追加するファイル名を入力してください");
   203	  sendbuf();
   204	  recvbuf();
   205	  if(strcmp(recv_buf,"can't open file")==0) return 0;
   206	  dent(recv_buf);
   207	  strcat(file,recv_buf);
   208	  fp = fopen(file,"w");
   209	  sprintf(send_buf,"ok");
   210	  sendbuf();
   211	  // データ受取スタート
   212	  recvbuf();
   213	  while(strcmp(recv_buf,"EOF")!=0){
   214	    fprintf(fp,"%s",recv_buf);
   215	    sendbuf();
   216	    recvbuf();
   217	  }
   218	  fclose(fp);
   219	  return 0;
   220	}
   221	
   222	int new_account(){
   223	  int i;
   224	  char buf[100]="./account/";
   225	  sprintf(send_buf,"作成するアカウント名を入力してください");
   226	  sendbuf();
   227	  recvbuf();
   228	  dent(recv_buf);
   229	  strcat(buf,recv_buf);
   230	  i=mkdir(buf,0777);
   231	  if(i==0)sprintf(send_buf,"アカウントを作成しました");
   232	  else sprintf(send_buf,"このアカウントは使えません");
   233	  sendbuf();
   234	  recvbuf();
   235	  return i;
   236	}
   237	
   238	int log_in(){
   239	  DIR *dirp;
   240	  struct dirent *p;
   241	  int find=0;
   242	
   243	  sprintf(send_buf,"アカウント名を入力してください");
   244	  sendbuf();
   245	  recvbuf();
   246	  dent(recv_buf);
   247	  dirp = opendir("account");
   248	  while((p = readdir(dirp)) != NULL){
   249	    if(strcmp(p->d_name,recv_buf)==0) find=1;
   250	  }
   251	  if(find==0) sprintf(send_buf,"アカウント名が間違っています");
   252	  else{
   253	    sprintf(send_buf,"ログイン:%s",recv_buf);
   254	    accountflag=1;
   255	    strcpy(accountname,recv_buf);
   256	  }
   257	  sendbuf();
   258	  recvbuf();
   259	  return 0;
   260	}
   261	
   262	int log_out(){
   263	  accountflag=0;
   264	  bzero(accountname,sizeof(accountname));
   265	  sprintf(send_buf,"ログアウトしました");
   266	  sendbuf();
   267	  recvbuf();
   268	  return 0;
   269	}
   270	
   271	//////////// 以下main部 /////////////////////
   272	
   273	void check(int i){
   274	  printf("check%d\n",i);
   275	}
   276	
   277	void delete_child(){
   278	  while(waitpid(-1,NULL,WNOHANG)>0);
   279	  signal(SIGCHLD,delete_child);
   280	}
   281	
   282	int main(int argc,char *argv[]){
   283	  int writer_len;
   284	  struct sockaddr_in reader_addr; 
   285	  struct sockaddr_in writer_addr;
   286	  char recv_msg[1024];
   287	  char send_msg[1024];
   288	  char buf[1024];
   289	  int i=0;
   290	  int pid;
   291	  pid_t ppid,cpid;
   292	  int pnum=4000;
   293	  int bindi=0;
   294	  int eflag=0;
   295	
   296	  //if(argc>1) pnum = atoi(argv[1]);
   297	  if(argc==2) printflag = atoi(argv[1]);
   298	
   299	  while(1){
   300	
   301	  /* ソケットの生成 */
   302	
   303	  if ((sockfd = socket(AF_INET, SOCK_STREAM, 0)) < 0) {
   304	    perror("reader: socket");
   305	    exit(1);
   306	  }
   307	
   308	  /* 通信ポート・アドレスの設定 */
   309	
   310	  bzero((char *) &reader_addr, sizeof(reader_addr));
   311	  reader_addr.sin_family = AF_INET;
   312	  reader_addr.sin_addr.s_addr = htonl(INADDR_ANY);
   313	  reader_addr.sin_port=htons(4000);
   314	
   315	    pnum=4000;
   316	    while(bind(sockfd,(struct sockaddr *)&reader_addr,sizeof(reader_addr))==-1){
   317	      pnum++;
   318	      reader_addr.sin_port=htons(pnum);
   319	    }
   320	    printf("pnum %d\n",pnum);
   321	
   322	    /* コネクト要求をいくつまで待つかを設定 */
   323	     if (listen(sockfd, 5) < 0) {
   324	      perror("reader: listen");
   325	      close(sockfd);
   326	      exit(1);
   327	    }
   328	     signal(SIGCHLD,delete_child);
   329	    /* コネクト要求を待つ */
   330	    writer_len = sizeof(struct sockaddr);
   331	    if ((new_sockfd = accept(sockfd,(struct sockaddr *)&writer_addr, &writer_len)) < 0) {
   332	      printf("reader: accept\n");
   333	      exit(1);
   334	    }
   335	    pid=fork();
   336	    if(pid==0)break;
   337	  }
   338	  close(sockfd);
   339	
   340	  dfps = fopen("debugs.txt","w");
   341	  accountflag=0;
   342	  bzero(accountname,sizeof(accountname));
   343	
   344	  while(1){
   345	    while(1){
   346	      if(accountflag==0){
   347		sprintf(send_buf,"\nコマンドを入力してください\n");
   348		strcat(send_buf,"practice\nread\nnew account\nlog in\nquit\n");
   349		sendbuf();
   350		recvbuf();
   351		if(strcmp(recv_buf,"practice\n")==0)break;
   352		else if(strcmp(recv_buf,"read\n")==0)read_file();
   353		else if(strcmp(recv_buf,"new account\n")==0)new_account();
   354		else if(strcmp(recv_buf,"log in\n")==0)log_in();
   355		else if(strcmp(recv_buf,"quit\n")==0){
   356		  eflag=1;
   357		  break;
   358		}
   359	      }else{
   360		sprintf(send_buf,"\nコマンドを入力してください\n");
   361		strcat(send_buf,"practice\nread\nreview\nlog out\nquit\n");
   362		sendbuf();
   363		recvbuf();
   364		if(strcmp(recv_buf,"practice\n")==0)break;
   365		else if(strcmp(recv_buf,"read\n")==0)read_file();
   366		else if(strcmp(recv_buf,"log out\n")==0)log_out();
   367		else if(strcmp(recv_buf,"review\n")==0){
   368		  rflag=1;
   369		  break;
   370		}else if(strcmp(recv_buf,"quit\n")==0){
   371		  eflag=1;
   372		  break;
   373		}
   374	      }
   375	    }
   376	    if(eflag)break;
   377	    if(rflag==0){
   378	      bzero(send_buf,sizeof(send_buf));
   379	      sprintf(send_buf,"\n練習ファイル一覧\n");
   380	      if(file_name(buf,"./read",".csv")) exit(1);
   381	      sprintf(send_buf,"%s%s",send_buf,buf);
   382	      printf("%s\n",buf);
   383	      sprintf(buf,"\nファイル名を入力してください");
   384	      strcat(send_buf,buf);
   385	      sendbuf();
   386	      recvbuf();
   387	      dent(recv_buf);
   388	    }
   389	    if(read_data(recv_buf))practice();
   390	    bzero(tango,sizeof(tango));
   391	    rflag=0;
   392	  }
   393	  sendbuf();
   394	  close(new_sockfd);
   395	  fclose(dfps);
   396	}  
\end{verbatim}
}
\end{document}
