\documentclass[a4j]{jarticle}
\usepackage{graphicx}

\title{コンパイラ実験 期末レポート}

\author{氏名: 木下直樹\\学籍番号: 09425521}

\date{提出日: 2015月7月27日\\締切日: 2015年11月26日}

\begin{document}
\maketitle

%%%%%%%%%%%%%%%%%%%%%%%%%%%%%%%%%%%%%%%%%%%%%%%
\section{クライアント,サーバモデルの通信の仕組みについて}
%%%%%%%%%%%%%%%%%%%%%%%%%%%%%%%%%%%%%%%%%%%%%%%

%%%%%%%%%%%%%%%%%%%%%%%%%%%%%%%%%%%%%%%%%%%%%%%
\section{自由課題プログラムの作成方針}
%%%%%%%%%%%%%%%%%%%%%%%%%%%%%%%%%%%%%%%%%%%%%%%
今回の自由課題では英単語暗記プログラムを作成する.

具体的には, サーバから単語の日本語訳をクライアントへ送信し, 英語訳をクライアント側の端末で入力してサーバに送信する. サーバはその正誤を判断し, 結果をクライアントへ送る.

また, 一連の処理に加え, 問題ファイルを追加する機能と, アカウント機能を追加する. 
このアカウント機能はログインしたクライアントが間違った問題を出力するためのものである.


%%%%%%%%%%%%%%%%%%%%%%%%%%%%%%%%%%%%%%%%%%%%%%%
\section{プログラムについて}
%%%%%%%%%%%%%%%%%%%%%%%%%%%%%%%%%%%%%%%%%%%%%%%
\subsection{プログラムの処理の流れ}
\subsubsection{初期状態の処理}
プログラムを実行すると,次の様にサーバからのメッセージが届く.
\begin{verbatim}
コマンドを入力してください
practice
read
new account
log in
quit
\end{verbatim}
この状態では practice, read, new account, log in, quit の5つのコマンドが入力できる.
それぞれの入力に対する処理をサーバが行う.
また, それ以外の入力をサーバに送信した場合, サーバは上記のメッセージを送信する処理部へループし, クライアント側のプログラムは上記のメッセージを受け取る処理部へループする.

クライアントプログラムの処理の流れは以下である.
\begin{verbatim}
(ループ){
   (ループ){
      (ループ){
         コマンドの入力
         上記の5つのコマンドのいずれかの入力でループから離脱
         quit のコマンドでプログラム終了
    }
     read, new account, log in, quitの処理
     コマンド入力時にpractice を入力した場合ループ離脱
   }
   practice の処理
}
\end{verbatim}
read, new account, log in, quit のコマンド処理は一塊のループで行い, practice はそのループの外で処理をする.

サーバプログラムの場合はクライアントからのコマンドを受け取ると, 正しいコマンドである場合その処理をする関数へとび, 正しくないコマンドを受け取った場合, クライアントがループして初期メッセージを受け取れるようにこちらもループをして再度初期メッセージを送信する. ただしquit コマンドは受け取ると関数に入ること無くサーバのプロセスが落ちる.

サーバプログラムの処理の流れは以下である.
\begin{verbatim}
(ループ){
   (ループ){
      コマンド入力要求
    正しいコマンドでループ離脱
      quit のコマンドでプロセスが落ちる
   }
   各コマンドの処理関数へ入る
}
\end{verbatim}


\subsubsection{log in後の処理}
log in コマンドを実行するとログインフラグの値を変更し, ログイン名を格納する配列にその文字列を格納する. 
また, コマンド入力要求が以下に変わる.
\begin{verbatim}
コマンドを入力してください
practice
read
review
log out
quit
\end{verbatim}
この状態でpractice を実行すると, 間違えた問題がaccountディレクトリ内にあるアカウントディレクトリ内のmislog.csvファイルへ出力する. 
log out で初期状態のコマンド要求と同じ処理へ戻る.

\subsection{プロトコルについて}
サーバとクライアント間のプロトコルで決められたコマンドとそのコマンドに対するサーバの挙動について説明する.
\begin{itemize}
\item practice
戻り値:文字列startの送信
\item read



\end{itemize}
%%%%%%%%%%%%%%%%%%%%%%%%%%%%%%%%%%%%%%%%%%%%%%%
\section{プログラムの使用法}
%%%%%%%%%%%%%%%%%%%%%%%%%%%%%%%%%%%%%%%%%%%%%%%

%%%%%%%%%%%%%%%%%%%%%%%%%%%%%%%%%%%%%%%%%%%%%%%
\section{プログラムの作成過程に関する考察}
%%%%%%%%%%%%%%%%%%%%%%%%%%%%%%%%%%%%%%%%%%%%%%%

%%%%%%%%%%%%%%%%%%%%%%%%%%%%%%%%%%%%%%%%%%%%%%%
\section{得られた結果に関する考察}
%%%%%%%%%%%%%%%%%%%%%%%%%%%%%%%%%%%%%%%%%%%%%%%

\end{document}
