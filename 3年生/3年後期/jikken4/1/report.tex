\documentclass[a4j]{jarticle}
\usepackage{graphicx}

\title{画像処理 レポート}

\author{氏名: 木下直樹\\学籍番号: 09425521}

\date{提出日: 2015月11月16日}

\begin{document}
\maketitle

\section{ppm画像を受け取りjpg画像を出力するプログラム}
入出力のファイル名を実行する際の引数で指定できるようにした.
\begin{verbatim}
例: ./a.out file1.ppm file2.jpg 
\end{verbatim}
以上の実行でfile1.ppmを受け取り,file2.jpgを出力できる.
\section{jpg画像を受け取りppm画像を出力するプログラム}
入出力のファイル名を実行する際の引数で指定できるようにした.
\begin{verbatim}
例: ./a.out file1.jpg file2.ppm
\end{verbatim}
以上の実行でfile1.jpgを受け取り,file2.ppmを出力できる.
\section{jpg画像から赤,緑,青の色を取り出し,それぞれ出力するプログラム}
実行する際の第二引数で入力ファイル名を指定し,出力するファイルは
第三引数にR,G,Bいずれかのアルファベットと.jpgを加えたファイル名を出力するようにした.
\begin{verbatim}
例: ./a.out file1.jpg file2
\end{verbatim}
以上の実行でfile1.jpgを受け取りfile2R.jpg,file2G.jpg,file2B.jpgを出力する.
出力例を以下に示す.\\
\begin{figure}
\begin{tabular}{cc}
\begin{minipage}{0.5\hsize}
\includegraphics[bb=0 0 768 576,scale=.3]{sangen.jpg}
\caption{file1.jpg}
\end{minipage}
\begin{minipage}{0.5\hsize}
\includegraphics[bb=0 0 768 576,scale=.3]{fileR.jpg}
\caption{file2R.jpg}
\end{minipage}
\end{tabular}
\begin{tabular}{cc}
\begin{minipage}{0.5\hsize}
\includegraphics[bb=0 0 768 576,scale=.3]{fileG.jpg}
\caption{file2G.jpg}
\end{minipage}
\begin{minipage}{0.5\hsize}
\includegraphics[bb=0 0 768 576,scale=.3]{fileB.jpg}
\caption{file2B.jpg}
\end{minipage}
\end{tabular}
\end{figure}
\end{document}
