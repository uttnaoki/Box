\documentclass[a4j]{jarticle}
\usepackage{graphicx}
\usepackage[left=25truemm,right=25truemm]{geometry}

\title{画像処理 レポート}

\author{氏名: 木下直樹\\学籍番号: 09425521}

%\date{提出日: \today}

\begin{document}
\maketitle

%%%%%%%%%%%%%%%%%%%%%%%%%%%%%%%%%%%%%%%%%%%%%%%%%%
\section{概要}
%%%%%%%%%%%%%%%%%%%%%%%%%%%%%%%%%%%%%%%%%%%%%%%%%%
パノラマ画像の自動生成を行うには,両方の画像の同じ物体上に出現している特徴点を自動で探し出す必要がある.
この際,第1画像の第i特徴点と第2画像の第j特徴点が似ているかどうかを判定する.
このための最も簡単な類似度尺度はSum of Squared Differencesである.
即ち,特徴点周辺の小領域の画像を高次元ベクトルで表し,高次元空間のユークリッドノルムで画像の類似度を測る.
類似度の高い特徴点対を4組以上選ぶことで,正しい変換行列の算出が可能となる.

%%%%%%%%%%%%%%%%%%%%%%%%%%%%%%%%%%%%%%%%%%%%%%%%%%%
\section{greedy.cの作成}
%%%%%%%%%%%%%%%%%%%%%%%%%%%%%%%%%%%%%%%%%%%%%%%%%
特徴点の対応付けプログラムgreedy.cを作成する.
このプログラムには, TKfilter.cで得た特徴点座標を使用するため, 
これをgreedy.cで使用するためにファイルへ出力するコードを追加しておく.

準備として, 出力された二つの画像とそれらの画像の特徴点座標を用い, 一方の画像の特徴点一点が他方の画像の特徴点全ての点それぞれに対して特徴点が似ているかどうかの計算を一方の画像の特徴点全てに実行し, それを行列の構造体へ格納する.

行列式は以下のようになる. (ただし, double型の値を100000で割, int型にキャストした値を表示するため, 実際に使用した値とは異なる)
{\baselineskip 2mm
  \scriptsize
\begin{verbatim}
  56  84  73  79  88  78  68  64  59  67  45  72  77  72  71  77  69  80  61  88  75  90  75  47  57  38  75  60  71  58
  53  84  70  71  60  62  58  62  70  54  61  73  63  65  69  68  65  43  47  78  58  69  88  51  44  47  70  60  64  38
  78  83  11  51  85  41  35  54  64  52  59  50  83  52  66  55  38  60  52  90  49  52  63  68  43  51  93  47  40  48
  66  80  49  46  45  51  40  60  49  37  60  39  93  48  57  56  53  63  44  75  44  45  44  61  42  45  80  40  54  43
  89 114  91  74   7  80  71  80  71  33 100  69 114  60  74  71  98  69  59  92  56  60  81  85  64  75  99  73  86  55
  78  99  35  10  70  49  22  71  71  46  66  57 110  58  76  66  64  70  65 111  51  49  73  84  53  64 106  53  57  40
  69  75  40  44  68  41  43  65  41  56  58  44  94  50  71  48  53  63  53  73  36  44  46  56  45  51 101  30  42  45
  72 102  40  28  53  47  13  62  61  41  69  57 102  59  73  67  67  67  61  97  45  54  68  71  52  56  89  60  64  33
  78  78  42  51  72  13  36  50  54  65  62  69  73  48  53  61  59  52  52  50  35  57  73  66  45  56  77  46  38  36
  74  74  53  60  43  63  44  64  59  16  56  21  95  47  68  52  47  56  39  84  41  46  56  49  37  44  90  47  51  44
  85  64  54  65  60  67  52  76  65  32  59   2 110  43  73  49  48  54  46  85  39  39  58  52  31  54 102  38  48  59
  63  42  76 102 107  67  88  82  66  81  30  70  54  49  68  63  50  50  42  47  56  82  73  37  58  39  63  43  56  66
  64  69  65  61  72  51  64  57   9  62  57  63 109  45  64  42  82  78  70  66  43  48  37  52  46  67 130  32  47  59
  51  64  65  75  90  45  58  46  46  76  47  80  48  68  48  67  63  66  44  46  49  81  64  40  61  41  58  49  50  51
  86  71  62  60  71  66  72  99  52  60  71  49 144  33  89  33  82  67  74  88  41  29  44  86  46  87 157  30  53  75
  51  56  59  77  85  58  58  43  62  56  38  62  46  62  47  63  38  51  35  66  49  66  66  33  40   5  48  47  47  38
  84  67  97  98  74  67  83  63  78  73  80  84  63  57  53  74  72  58  43  50  48  79  91  60  65  58  61  60  62  59
  63  58  65  61  82  68  74  78  48  52  54  51 114  46  74  21  80  65  59  86  43  32  39  56  45  78 141  27  51  68
  84  60 101 121 101  55 102  74  67 102  74 100  66  56  63  85  79  72  52   8  59  95  85  59  79  64  60  61  62  74
  87  75  98 131 140  77 108  68 105 114  68 128   6  97  62 109  62  69  45  69  87 125 132  63  86  53  45  82  73  73
  63  51  49  50  53  54  47  52  41  37  42  33  90  38  46  33  41  40  52  77  30  33  54  39   8  40  87  34  37  45
  75  47  44  62  80  34  61  52  44  54  53  46  76  37  44  30  41  40  39  57  29  30  58  50  29  52  91  22   4  55
  54  65  48  51  54  59  48  58  44  34  51  31  98  46  65  27  51  45  49  91  35  29  47  46  24  42 103  34  43  40
  53  37  77  97  91  64  79  65  59  69  38  61  48  63  46  56  49  51  40  49  50  75  71  19  44  34  61  46  47  64
  52  71  67  69  32  60  60  51  66  29  57  61  56  52  50  56  53  47  25  60  44  57  66  49  46  34  51  51  56  34
  69  40  45  54  85  49  54  63  51  55  35  46  77  46  41  46  41  54  55  74  36  39  67  41  18  50  95  37  31  57
  71  47  50  59  76  49  67  70  42  53  54  45  99  30  57  24  67  52  46  61  30  26  43  52  39  70 122  17  28  64
  65  81  40  43  58  40  34  49  57  45  57  60  64  53  60  67  40  60  38  73  43  62  67  54  44  35  62  53  53  11
  60  57  39  37  60  48  36  60  43  40  37  35  93  37  50  34  44  47  55  80  29  29  46  46  18  42  94  32  37  43
  66  74  47  63  72  38  41  48  61  55  58  57  60  56  56  69  40  53  37  59  48  64  66  54  42  27  49  47  43  29

\end{verbatim}
}
第一画像の第(行数)番目の特徴点と第二画像の第(列数)番目の特徴点のSSD(差の二乗和)が30点$×$30点$=$900点格納される. 
この行列の値は小さいものほど対応する特徴点二点が一致している可能性が高いといえる. 
そのためgreedy.cでもっとも重要とされるのはこの行列中の値で小さいものを探索し, 第一画像の特徴点と第二画像の特徴点をうまく対応させる処理である. 
その処理として, 以下の様な方法を実装した.

\begin{itemize}
\item{方法1}

第一画像で特徴点らしさの高い点から第二画像の特徴点との対応を決定する. 
処理手順としては, 上記の様な行列の一番上の行から一行ずつその行の中で一番値の小さな点を探索し, 第一画像の行数番目の特徴点と第二画像の列数番目の特徴点が対応すると決定する. 

行列の決定した点の列の値に無限値を格納し, それ以降の探索で同じ点を採用しないようにする. 
実行結果の特徴点の対応セットは30セット中12セットが一致していた.
特徴点をより正確に一致させるために次の様な方法を採用した.

\item{方法2}

行列の中の値全てを探索し, 一番値の小さい値の対応点を決定とする.
対応点が決まるとその点の行と列の値に無限値を格納し, それ以降の処理で同じ点を取らない様にする. 
実行結果は30セット中19セットが実際に一致していた. 
\end{itemize}

\section{問題と改善案}
今回作成したgreedy.cでは方法1で取り組んだ場合も方法2で取り組んだ場合も現状30個の特徴点を全て正確に対応させられていない.
しかし, この問題はgreedy.cだけでの問題ではない. greedy.cの入力ファイルである0.feaと2.feaに記された座標はそれぞれ30点であるが, 一方の特徴点が他方の画像中にあるとは限らないのである. そのため, いくらgreedy.cの性能を向上させても30点全てを正確に対応させることは物理的に不可能なのである. 

\begin{itemize}
\item{改善案1}

TKfilter.cから出力される特徴点を100個にするなど, 余分に多く特徴点を出力する. 
これにより, 多くの特徴点の中から対応する特徴点のセットを考えることができ, greedy.cで指定する特徴点セットの数が少ないほどその指定した数の特徴点セットの信頼性が高くなり, また今回の30点で試行する場合でも方法2よりもよい結果になることが期待できる. 
しかし, 現状のgreedy.cの処理では方法1も方法2も処理が重いため, 行列の大きさが飛躍的に大きくなってしまうこの改善案を採用するにはgreedy.cの処理効率を向上させなければならない.

\item{改善案2}

TKfilter.cから出力される一方の画像の特徴点座標を他方の画像中にあるもののみとする条件を加える. 
これによりgreedy.cが捌く行列は大きくならないため, 処理の軽量化が期待できる. 
ただしこの方法では, 一方の画像中の点が他方の画像中にあるかないかを判断できることが前提である. 

\end{itemize}
\end{document}
